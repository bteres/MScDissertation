% Appendix Template

\chapter{Approximation to the Heidler Function - Development} % Main appendix title

\label{AppendixDev} % Change X to a consecutive letter; for referencing this appendix elsewhere, use \ref{AppendixX}

\lhead{\chaptername~\thechapter. \emph{Approximation Development}} % Change X to a consecutive letter; this is for the header on each page - perhaps a shortened title

%----------------------------------------------------------------------------------------
%    Overview
%----------------------------------------------------------------------------------------

\section{Overview}
\label{sec:app_dev_overview}
The process of developing an approximation to the Heidler function is easily described in a few steps. This appendix goes into more detail and shows all the necessary steps in developing the approximation.

%----------------------------------------------------------------------------------------
%    Developing the Approximation
%----------------------------------------------------------------------------------------

\section{Developing the Approximation}
\label{sec:app_dev_developing_the_approximation}
\eqnref{eqn:app_HFrise} shows the Heidler rise time function and it is clear that this function cannot be analytically integrated.
\begin{equation}
    x_h \left( t \right) = \frac{{\left (\frac{t}{\tau_1} \right )}^{n_h}}{1 + {\left (\frac{t}{\tau_1} \right )}^{n_h}}
    \label{eqn:app_HFrise}
\end{equation}
A plot of this is seen in \figref{fig:AppHeidlerFunctionRise} (an S-curve).
\inputtikzfig{HeidlerFuncRise}{AppHeidlerFunctionRise}{Graph depicting the rise function of the Heidler function (an S-curve).}

The S-curve rises to a value of one and remains constant. It is clear from this that a step response is required. The approximation rise time function can be defined as in \eqnref{eqn:app_approx_rise_laplace}.
\begin{equation}
    X_a \left( s \right) = \frac{1}{s} H \left( s \right)
    \label{eqn:app_approx_rise_laplace}
\end{equation}
Where: \\
\begin{tabular}{cll}
    $H \left( s \right)$ & = & Transfer function (Laplace domain) \\
    $\frac{1}{s}$ & = & Unit step function in the Laplace domain
\end{tabular}\\
The step response of an n-th order real and negative pole in the Laplace domain produces an S-shaped curve response in the time domain. Therefore the transfer function can be defined as in \eqnref{eqn:app_tf_func}.
\begin{equation}
    H \left( s \right) = \frac{1}{\left( \frac{s}{\omega_0} + 1 \right)^{n_a}}
    \label{eqn:app_tf_func}
\end{equation}
Where: \\
\begin{tabular}{cll}
    $\omega_0$ & = & Some real and negative value
\end{tabular}\\\\
By substitution, \eqnref{eqn:app_approx_rise_laplace} becomes \eqnref{eqn:app_approx_rise_laplace_final}.
\begin{equation}
    X_a \left( s \right) = \frac{1}{s} \frac{1}{\left( \frac{s}{\omega_0} + 1 \right)^{n_a}}
    \label{eqn:app_approx_rise_laplace_final}
\end{equation}

The time domain equation is required which can be found by taking the inverse Laplace transform of \eqnref{eqn:app_approx_rise_laplace_final} as defined in \eqnref{eqn:app_approx_rise}.
\begin{align}
    x_a \left( t \right) & = \mathcal{L}^{-1} \left \{ X_a \left( s \right) \right \} \notag\\
     & = \mathcal{L}^{-1} \left \{ \frac{1}{s} \frac{1}{\left( \frac{s}{\omega_0} + 1 \right)^{n_a}} \right \} \notag\\
     & = 1 - e^{-\omega_0 t} \left ( \sum\limits_{i=0}^{n_a} \frac{\omega_0^i t^i}{i!} \right )
    \label{eqn:app_approx_rise}
\end{align}

By substituting this back into the generalised form of the lightning stroke current shown in \eqnref{eqn:app_eqn_small}, the complete approximation can be seen in \eqnref{eqn:app_approx}.
\begin{align}
    i_a \left( t \right) & = \frac{I_0}{\eta} x_a \left( t \right) y \left( t \right) \label{eqn:app_eqn_small} \\
    & = \frac{I_0}{\eta} \left ( 1 - e^{-\omega_0 t} \left ( \sum\limits_{i=0}^{n_a} \frac{\omega_0^i t^i}{i!} \right ) \right ) y \left( t \right) \notag\\
    & = \frac{I_0}{\eta} \left ( 1 - e^{-\omega_0 t} \left ( \sum\limits_{i=0}^{n_a} \frac{\omega_0^i t^i}{i!} \right ) \right ) e^{-t/\tau_2}
    \label{eqn:app_approx}
\end{align}

%----------------------------------------------------------------------------------------
%    Conclusion
%----------------------------------------------------------------------------------------

\section{Conclusion}
\label{sec:app_dev_conclusion}
This appendix provides all the mathematical steps required to develop the approximation to the Heidler function. This starts at the step response function and goes through until the final equation.
