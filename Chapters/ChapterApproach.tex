% -*- root: ../main.tex -*-
% Chapter Template

\chapter{Approach Taken} % Main chapter title

\label{ChapterApproach} % Change X to a consecutive number; for referencing this chapter elsewhere, use \ref{ChapterX}

\lhead{\chaptername~\thechapter. \emph{Approach Taken}} % Change X to a consecutive number; this is for the header on each page - perhaps a shortened title
\begin{quote}
An overview of the work addressed in this study is detailed in this chapter. The problem statement, contribution of the dissertation and the methodology of the study are provided. This gives a quick overview of why the study has been undertaken, what it provides and how it was performed and evaluated.
\end{quote}

%----------------------------------------------------------------------------------------
%    Problem Statement
%----------------------------------------------------------------------------------------

\section{Problem Statement}
\label{sec:approach_problem_statement}
The Heidler function was developed to represent any lightning current waveshape for modelling lightning currents in design or analysis \cite{Heidler2002}. However there is no analytical integral to the Heidler function. This leads to issues when trying to calculate the \gls{em} fields produced by a lightning stroke. Moreover, there is no way to analytically calculate the Fourier transform of the Heidler function and hence the frequency components of a lightning stroke. When designing \glspl{lps} and/or calculating the induced effects of lightning strikes, there is a need for the analytical integral of the lightning current waveshape.

The IEC~62305-1 standard defines the Heidler function as the standardised lightning current waveshape. As there is no integral to this function, many researchers have used the double exponential function in its place. There are several limitations associated with the double exponential function making it an unsuitable replacement. A key limitation is that there is an instantaneous rise in current at $t=0$ which is not physically realisable. \chapref{ChapterBackground} gives a more detailed background into the applications of lightning current models, the IEC~62305-1 and the different lightning models and approximations.

There is a requirement for a function that can be used in place of the Heidler function in the standard. This function should take a similar form to that of the Heidler function, it should be intuitive when compared to the Heidler function and the mathematics in using this function should be as simple as that of the Heidler function. Most importantly this function must have an analytical integral for lightning current applications.

%----------------------------------------------------------------------------------------
%    Contribution of this dissertation
%----------------------------------------------------------------------------------------

\section{Contribution of this Dissertation}
\label{sec:approach_contribution_of_this_dissertation}
This study develops a function that approximates the Heidler function in the time domain. This approximation has the advantage of having an analytical integral and taking the same form as that of the Heidler function; only the parts of the Heidler function that cannot be integrated are replaced in producing this approximation. This approximation is easy to use and the parameters for creating different lightning current waveshapes are determined easily. In a system design or simulation, it would be trivial to replace the Heidler function with this approximation. Most importantly, it can be utilised in performing Maxwell's equations or finding the Fourier transform (frequency components). \chapref{ChapterApprox} details the development process of this function with its properties.


%----------------------------------------------------------------------------------------
%    Scope and Limitations
%----------------------------------------------------------------------------------------

\section{Scope and Limitations}
\label{sec:approach_scope_and_limitations}
The approximation developed in this research has a clearly defined scope: it is based on the waveshapes mentioned in the IEC~62305-1 lightning protection standard. The approximation is ideally suited for use in cases where the integral of the Heidler function is required. Therefore, the IEC~62305-1 standard is used as a starting point for the research which introduces some limitations. For example, the parameters for the various \glspl{lpl} and lightning current models are dictated by those defined in this standard. The research is limited to the definitions in this standard and comparisons are only made to the two waveshapes mentioned therein. The parameters of the approximation are determined empirically from those defined in the IEC~62305-1 for the Heidler function. No evidence is provided in this research to show that this approximation can be used outside of the bounds of the IEC~62305-1. \chapref{ChapterDiscussion} discusses further research that can be carried out to either overcome or verify these limitations.


%----------------------------------------------------------------------------------------
%    Study Methodology
%----------------------------------------------------------------------------------------

\section{Methodology of the Study}
\label{sec:approach_study_methodology}
This study is performed by evaluating the limitations associated with the Heidler function and devising a solution. The approach taken in developing this approximation is different to the other approximations outlined in \secref{sec:background_approximations} because the development is done in the Laplace domain and the inverse Laplace transform is obtained. This is done only for the part of the equation that cannot be integrated. This part of the function is substituted back into the general form of the function to create the overall approximation. This is all discussed in more detail in \chapref{ChapterApprox}

As this study is the development of an equation, all results are based on simulations of the approximation. In order to determine the accuracy of the approximation, a control is required. The IEC~62305-1 details two waveshapes and gives the corresponding parameters for the Heidler function for these waveshapes. The approximation is simulated alongside these waveshapes and the maximum errors are obtained by calculating the absolute difference between the approximation and the Heidler function. The derivatives are compared in a similar manner. \chapref{ChapterResults} details all of the results obtained using this methodology.

Conclusions are drawn about the accuracy of the approximation and the frequency components are plotted as an indication of their similarity to the IEC~62305-1 standard. This is all detailed in \chapref{ChapterDiscussion}.

%----------------------------------------------------------------------------------------
%    Conclusion
%----------------------------------------------------------------------------------------

\section{Conclusion}
\label{sec:approach_conclusion}
This chapter has given an overview of the entire study. It has essentially answered the questions of what, why and how relating to the study. It has also given an outline to which chapters answer what questions and how.

The following chapter gives the background that relates to the work done. This is done by detailing some applications of lightning current models, the lightning protection standard, lightning current models and some of the work that has been done in approximating the Heidler function.
