% -*- root: ../main.tex -*-
% Chapter Template

\chapter{Approach Taken} % Main chapter title

\label{ChapterApproach} % Change X to a consecutive number; for referencing this chapter elsewhere, use \ref{ChapterX}

\lhead{\chaptername~\thechapter. \emph{Approach Taken}} % Change X to a consecutive number; this is for the header on each page - perhaps a shortened title
\begin{quote}
\note[BRT]{Chapter abstract goes here!}
\end{quote}

%----------------------------------------------------------------------------------------
%    Problem Statement
%----------------------------------------------------------------------------------------

\section{Problem Statement}
\label{sec:approach_problem_statement}
Lightning research has led to many advancements in \glspl{lps} design and analyses. The Heidler function is described (in the lightning protection standards) as the lightning current model that is used in developing \glspl{lps}. This however cannot be analytically integrated and hence the \gls{lemp} equations cannot be analytically calculated. This has an effect on lightning detection protection system designs and analyses.

Therefore, an approximation to the Heidler function with an analytical integral would provide researchers and designers with an means of calculating electric and magnetic fields of lightning currents. Moreover, it would be possible to obtain a true frequency spectrum of a lightning current, giving rise to further analyses of the individual frequencies of lightning currents. By creating a ``customisable'' lightning current model, any common (lightning or high voltage impulse) waveshape can be modelled for analyses.

An equation is designed to approximate the Heidler function that is as ``customisable'' and is trivial to use and implement in analyses. The result is used to simulate several common lightning current waveshapes and these are compared to the same waveshapes simulated by the Heidler function. Furthermore, the simulations are tested against the characteristics outlined in the IEC~62305 to determine the viability of the approximation as a replacement to the Heidler function when an integral or an accurate frequency spectrum is required.

%----------------------------------------------------------------------------------------
%    Approximation Function and Evaluation
%----------------------------------------------------------------------------------------

\section{Approximation Function and Evaluation}
\label{sec:approach_approximation_function_and_evaluation}

%-----------------------------------
%    Scope
%-----------------------------------
\subsection{Scope}
\label{sub:approach_scope}

%-----------------------------------
%    Evaluation
%-----------------------------------
\subsection{Evaluation}
\label{sub:approach_evaluation}


%----------------------------------------------------------------------------------------
%    Contribution of this dissertation
%----------------------------------------------------------------------------------------

\section{Contribution of this dissertation}
\label{sec:approach_contribution_of_this_dissertation}

%----------------------------------------------------------------------------------------
%    Conclusion
%----------------------------------------------------------------------------------------

\section{Conclusion}
\label{sec:approach_conclusion}
\ldots The following chapter \ldots
