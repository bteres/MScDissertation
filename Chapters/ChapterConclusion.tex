% -*- root: ../main.tex -*-
% Chapter Template

\chapter{Conclusion} % Main chapter title

\label{ChapterConclusion} % Change X to a consecutive number; for referencing this chapter elsewhere, use \ref{ChapterX}

\lhead{\chaptername~\thechapter. \emph{Conclusion}} % Change X to a consecutive number; this is for the header on each page - perhaps a shortened title

An approximation to the Heidler function that has an analytical integral has been developed and discussed. This is useful in situations where the \gls{em} fields produced by a lightning stroke need to be calculated. It is also useful in any scenario where the frequency components of a lightning stroke are required for evaluation. The properties of the approximation are discussed with reference to the IEC~62305-1 standard. The viability of the approximation as a suitable replacement for the Heidler function in the standard have been evaluated through the investigation of the simulations produced. As the study is based on the guidelines in the IEC lightning protection standard, the lightning strokes defined therein are the ones used for the evaluation of viability.

It can be seen that the waveshapes produced by the approximation are very similar to those produced by the Heidler function. The maximum error in amplitude for the first and subsequent lightning stroke currents is less than 1.5\%. The maximum error in the derivative is however greater but still less than 8\%. This is still within the parameters defined in the standard which are based on some of the original lightning current analyses. The simulations detail the frequency response of the approximation for both waveshapes. There is no way of quantifying the error in this result because the Heidler function has no analytical integral and hence no Fourier transform. However the results are similar to what is expected in the lightning protection standard. All of this provides evidence that this approximation is a suitable replacement for the Heidler function when integrals and frequency spectra of lightning strokes are required.

The approximation has been designed using the Heidler function as a base and therefore they are easily interchangeable. When breaking the functions into components, the only difference is in the rise time functions. Other than that the two equations contain the same parameters. There is a ratio for the rise time constant that is found for different waveshapes of the Heidler function. The inverse of this ratio holds true for the approximation giving further evidence of the consistency across the two functions. Therefore the approximation can easily be used in place of the Heidler function taking into account the quantified error (if necessary).

Additional work is required to optimise the approximation. This would further reduce the error and make working with the approximation even easier. This work includes proving or disproving the hypothesis that the error remains constant for any waveshape produced by the approximation. There is only an empirical optimisation of the parameters used in this study. A full optimisation algorithm should be run in order to reduce the 1.5\% inaccuracy between the approximation and the Heidler function. It is posited that there may be some relationship between the parameters used in the Heidler rise time function and the approximation rise time function. Finding this relationship would further simplify the use of the approximation in studies and system design.

The approximation that is developed is found to be a suitable replacement for the Heidler function with the errors quantified and an analytical integral. Evidence is provided to show that the approximation to the Heidler function developed in this dissertation, can be used for the first and subsequent short strokes mentioned in the IEC~62305-1 with less than 1.5\% error.
