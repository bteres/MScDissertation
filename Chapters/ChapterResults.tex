% -*- root: ../main.tex -*-
% Chapter Template

\chapter{Results: A Comparison to the Heidler Function} % Main chapter title

\label{ChapterResults} % Change X to a consecutive number; for referencing this chapter elsewhere, use \ref{ChapterX}

\lhead{\chaptername~\thechapter. \emph{Results: A Comparison to the Heidler Function}} % Change X to a consecutive number; this is for the header on each page - perhaps a shortened title

\begin{quote}
This chapter shows the results obtained by simulating the approximation and comparing the simulations to those of the Heidler function. In order for the results to be understood, the experimental methodology is first defined which explains how the simulations are carried out and how the comparisons are made. Both the first short stroke and the subsequent short stroke (10/350 and 0.25/100 respectively) are analysed. In the analyses, the functions and their derivatives are compared to the Heidler function and the current densities of the approximation are plotted.
\end{quote}

%----------------------------------------------------------------------------------------
%   Overview
%----------------------------------------------------------------------------------------

\section{Overview}
\label{sec:results_overview}
As mentioned in \chapref{ChapterApproach}, the results obtained in this study are simulated. The simulations used in this study are those of the short lightning current waveshapes detailed in the IEC~62305-1 standard \cite{IEC623051}. These are the initial and subsequent short strokes (10/350 and 0.25/100 respectively). This chapter details the experimental methodology and compares the waveshapes obtained using the Heidler function (\eqnref{eqn:HF}) with those obtained using the approximation (\eqnref{eqn:approx}).

%----------------------------------------------------------------------------------------
%    Experimental Methodology
%----------------------------------------------------------------------------------------

\section{Experimental Methodology}
\label{sec:results_experimental_methodology}
The requirement in this study is to determine the accuracy of the approximation to the Heidler function. This is achieved by running mathematical simulations using mathematical modelling software such as MATLAB\textsuperscript{\textregistered} \cite{MATLAB}, Mathematica\textsuperscript{\textregistered} \cite{mathematica}, Maxima \cite{maxima}, etc. In order to determine a level of accuracy, the values used in the IEC~62305-1 are used as a control. The parameters used in creating the Heidler function are detailed in Table B.1 of the IEC~62305-1 standard. These values are for the 10/350 and the 0.25/100 waveshapes (first and subsequent short strokes respectively).

Initial values are estimated for the parameters of the approximation from the Heidler function parameters mentioned above. These values are then empirically optimized in order to minimize the error between the approximation and the Heidler waveshapes. These tabulated and calculated parameters are utilised in \eqnrefs{eqn:HF}{eqn:approx} respectively to evaluate the accuracy of the approximation. An $n_a$ of 33 is found to be appropriate for the waveshapes defined in the standard which have an $n_h$ of 10 (see \chapref{ChapterDiscussion} for more).

There are various peak current values for the different \glspl{lpl} defined in Table B.1 in the standard \cite{IEC623051}. However, as the peak current is only determined by $I_0$ (peak current) and $\eta$ (peak current correction), the peak current values have no effect on the waveshapes defined in \eqnrefs{eqn:HF}{eqn:approx} or their respective errors.

The evaluation includes three simulations per waveshape. These are the current waveshape, the change in current (first derivative) and the current density (Fourier transform).

The current waveshapes of the Heidler function and the approximation are plotted with the required parameters. The absolute value of the difference between the two functions is determined and the maximum error is defined from this as a percentage of the Heidler function. The first derivative is evaluated in very much the same manner as the current waveshape.

The current density is evaluated by using the parameters calculated for the approximation in \eqnref{eqn:approx_fourier_freq}. This is then plotted on log-log axes. As there is no analytical Fourier transform of the Heidler function, this is purely an indication and no quantification of error can be obtained from this.

%----------------------------------------------------------------------------------------
%    First Short Stroke (10/350)
%----------------------------------------------------------------------------------------

\section{First Short Stroke (10/350)}
\label{sec:results_FS}
The first short stroke as defined by the IEC~62305-1 has a front time of 10~\usec and a decay time of 350~\usec (see \secref{sec:background_iec62305}) \cite{IEC623051}. A graph depicting both the Heidler function (solid line) and the approximation (dashed line) waveshapes can be seen in \figref{fig:FS}.
\inputtikzfig[t]{FS}{FS}{Graph of the first short stroke (10/350) current model using both the Heidler function and the approximation. The time scale is up to 200~\usec and the amplitude is as high as 210~kA.}
The values used in both of the equations to create the waveshapes seen in \figref{fig:FS} are shown in \tabref{tab:FS}.
\begin{table}[htbp]
    \centering
    \caption{Parameters used in \eqnrefs{eqn:HF}{eqn:approx} to plot the waveshapes shown in \figref{fig:FS}.}
    \begin{tabular}{lcc}
        \textbf{Parameter} & \textbf{Heidler} & \textbf{Approximation} \\
        \hline
        $I_0$ [kA] & 200 & 200 \\
        $\eta$ & 0.93 & 0.93 \\
        $n_a$ & - & 33 \\
        $n_h$ & 10 & - \\
        $\omega_0$ [rad/s] & - & 1 768 211 \\
        $\tau_1$ [\usec] & 19 & - \\
        $\tau_2$ [\usec] & 485 & 485
    \end{tabular}
    \label{tab:FS}
\end{table}

It is clear from the figure that the approximation closely follows the waveshape produced by the Heidler function. The error is quantified by determining the error as a function of time as seen in \eqnref{eqn:error}. The maximum error is found by equating the first time derivative of the error function to zero as in \eqnref{eqn:maxerror} and solving for $t$. This time is substituted back into the error function to obtain the maximum error in kA. This value is found as a percentage of the peak value of the Heidler function.
\begin{align}
    e \left( t \right) & = \left | i_a \left ( t \right ) - i_h \left ( t \right ) \right | \label{eqn:error} \\
    e' \left( t \right) & = 0 \label{eqn:maxerror}
\end{align}
Where: \\
\begin{tabular}{cll}
    $e \left( t \right)$ & = & Error function [A] \\
    $e' \left( t \right)$ & = & Derivative of error function [A/s]
\end{tabular}\\

In the case of the 10/350 waveshape, with the parameters defined in \tabref{tab:FS}, the maximum error is defined as \input{AdditionalFiles/FS/FSerror.txt}\unskip \%. This error is seen to occur during the rise part of the waveshape which is expected because the decay functions are identical.

The next comparison made is between the first derivatives of both the Heidler function and the approximation (\eqnrefs{eqn:dHF}{eqn:approx_deriv} respectively). This shows the difference in the instantaneous change in current of the two waveshapes. The same parameters are used, i.e. those in \tabref{tab:FS}. The graph showing both of these waveshapes can be seen in \figref{fig:dFS}.
\inputtikzfig{dFS}{dFS}{Graph of the first time derivative of the first short stroke (10/350) current model using both the Heidler function and the approximation. The time scale is up to 200~\usec and the amplitude is as high as 30~kA/\usec.}

The error is more pronounced in the derivative. Using the same method as above to obtain the maximum error as a percentage of the Heidler function maximum, the maximum error is calculated to be \input{AdditionalFiles/FS/FSdError.txt}\unskip \% (as before, the error is seen during the rise part of the waveshape).

The maximum dI/dt occurs during the rise time of the function. Another characteristic of the plot is that the exponential decay is much longer than the rise. This causes the negative component of the derivative to be much smaller but longer than the rise time component.

In both \figrefs{fig:FS}{fig:dFS}, the time scale goes up to 200~\usec. This is because the tails of the two waveshapes are the same. Therefore the resolution needs to be shown on the rise time of the waveshapes. The amplitudes shown in the two graphs are large enough to show the maximum values (200~kA in \figref{fig:FS} and 27.5~kA/\usec in \figref{fig:dFS}).

For completeness, the current density of the approximated first short stroke is plotted in \figref{fig:FreqFS}. This is produced using \eqnref{eqn:approx_fourier_freq} and the parameters in \tabref{tab:FS}. A current density is plotted because this is what is shown in Figure B.5 in the IEC~62305-1 standard.
\inputtikzfig{FreqFS}{FreqFS}{Current density of the approximation model produced from the waveshape plotted in \figref{fig:FS}.}

It is difficult to calculate an error in the case of the current density as there is no analytical solution to the integral and hence the Fourier transform of the Heidler function. Therefore any representation of this would be based on numerical methods with inherent errors.

%----------------------------------------------------------------------------------------
%    Subsequent Short Stroke (0.25/100)
%----------------------------------------------------------------------------------------

\section{Subsequent Short Stroke (0.25/100)}
\label{sec:results_SS}
As stated in \chapref{ChapterApproach}, the purpose of this study is to find an appropriate approximation to the Heidler function that can be used as a substitute when designing systems using the guidelines of the IEC~62305-1 standard. Therefore both the waveforms prescribed by the standard need to be analysed. Because of this, this section is similar to the last. However slightly different conclusions can be drawn from the two different waveshapes.

According to the standard, the subsequent short stroke has a front duration of 0.25~\usec and a decay time of 100~\usec \cite{IEC623051}. This implies a much sharper rise time than that of the first short stroke. Again, the three metrics shown here are the actual waveshape, the first time derivative and the current density.

\inputtikzfig{SS}{SS}{Graph of a subsequent short stroke (0.25/100) current model using both the Heidler function and the approximation. The time scale is up to 5~\usec and the amplitude is as high as 52~kA.}
A graph depicting both the Heidler function (solid line) and the approximation (dashed line) waveshapes can be seen in \figref{fig:SS}.
The values used in both of the equations to create the waveshapes seen in \figref{fig:SS} are shown in \tabref{tab:SS}.
\begin{table}[htbp]
    \centering
    \caption{Parameters used in \eqnrefs{eqn:HF}{eqn:approx} to plot the waveshapes shown in \figref{fig:SS}.}
    \begin{tabular}{lcc}
        \textbf{Parameter} & \textbf{Heidler} & \textbf{Approximation} \\
        \hline
        $I_0$ [kA] & 50 & 50 \\
        $\eta$ & 0.993 & 0.993 \\
        $n_a$ & - & 33 \\
        $n_h$ & 10 & - \\
        $\omega_0$ [rad/s] & - & 74 000 000 \\
        $\tau_1$ [\usec] & 0.454 & - \\
        $\tau_2$ [\usec] & 143 & 143
    \end{tabular}
    \label{tab:SS}
\end{table}

With the method outlined above, the maximum error is calculated to be \input{AdditionalFiles/SS/SSerror.txt}\unskip \% (as before, the error is seen during the rise part of the waveshape). The decay time is so much greater than the front duration (400$\times$), that there is almost no decay in the graph shown. This is so that the variation in the rise part of the function can be seen.

The first time derivative of both the Heidler function (solid line) and the approximation (dashed line) are shown in \figref{fig:dSS}.
\inputtikzfig{dSS}{dSS}{Graph of the first time derivative of a subsequent short stroke current model using both the Heidler function and the approximation. The time scale is up to 5~\usec and the amplitude is as high as 300~kA/\usec.}
Again the error in the first time derivative is more pronounced than in the actual waveshape and this error is calculated to be \input{AdditionalFiles/SS/SSdError.txt}\unskip \% (as before, the error is seen during the rise part of the waveshape). The decay time is so large in comparison to the rise time, that the negative dI/dt is negligible and in most engineering applications can be assumed to be zero. The maximum change in current is ten times greater than that of the first return stroke.

Once again, the current density of the approximated subsequent short stroke can be seen in \figref{fig:FreqSS}. This is produced using \eqnref{eqn:approx_fourier_freq} and the parameters in \tabref{tab:SS}. As expected, there are higher frequency components in the subsequent short stroke than in the first short stroke. However, the amplitude is lower.
\inputtikzfig{FreqSS}{FreqSS}{Current density of the approximation model produced from the waveshape plotted in \figref{fig:SS}.}

%----------------------------------------------------------------------------------------
%    Conclusion
%----------------------------------------------------------------------------------------

\section{Conclusion}
\label{sec:results_conclusion}
This chapter has discussed the experimental methodology. The results of the experiment are simulated and errors have been defined for the first and subsequent short strokes.

The following chapter utilises these results in order to draw conclusions about the viability of this approximation as a suitable replacement for the Heidler function in the IEC~62305-1 standard. Some comments about the further work that can be carried out are also made.
